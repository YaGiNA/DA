\cleardoublepage
\selectlanguage{english} 
\thispagestyle{plain}
\begin{center}
    \Large
    \textbf{Towards a Safer Social Media: AI-Powered Detection of Disinformation in Text and Audio}
        
    \vspace{0.4cm}
    \textbf{Yuta Yanagi}
       
    \vspace{0.9cm}
    \textbf{Abstract}
   
\end{center}
%\begin{center}
%    \vspace{5mm}
%        {\Large \bf Proposal of Artificial Intelligence Models for\\Detecting Disinformation on Diversified Social Media}\\
%    \vspace{10mm}
%    Yuta Yanagi
%\end{center} 
%\begin{abstract}
This doctoral dissertation examines innovative methodologies for identifying disinformation, 
with a particular emphasis on the evolving challenges posed by synthetic audio in social media. 
Two principal methodologies are put forth to address the shortcomings of existing text-centric approaches and the accelerated advancement of audio synthesis technology.
The initial method employs comment-generation techniques to augment the detection of disinformation on text-based social media platforms. 
The Grover model serves as a foundation for developing a system that generates comments based on news articles and existing comments, thereby simulating post-diffusion scenarios. 
The experiments demonstrate that the incorporation of generated comments enhances the recall of disinformation detection, 
thus enabling the identification of such content even when comment availability is limited.
The second method addresses the detection of synthetic audio conveying disinformation, 
with a particular focus on both the speech waveform and the content. 
A novel framework combines waveform analysis using established spoofing detection models (RawNet2, RawBoost, SSL-Antispoofing)
with content analysis leveraging text embeddings from the MuMiN dataset. 
This approach demonstrates a notable enhancement in the classification of synthetic voices propagating disinformation when compared to a strategy that relies exclusively on waveform analysis.

%\end{abstract}

%Recently, it is expected that personal information stored by different service providers are combined securely and it will create a new service. However, there is a risk that a specific user record can be identified by the combined personal information, and the user's sensitive information is revealed. Also, the personal information collected by the service provider must not be disclosed to other service providers because of security and privacy issues. Thus, related researches have been conducted on distributed anonymization methods, which combine the personal information stored by the providers and sanitize it to ensure a policy of anonymity with the minimum disclosure.

%However, in those researches, if sets of the users among the providers are different, a problem occurs that the users' presence in either provider may be revealed. Therefore, this paper proposes a new indicator, named {\it $\delta$-site-presence}, which represents the probability of the users' presence being revealed. Also, this paper proposes an improved distributed anonymization protocol which satisfies the proposed indicator. This protocol uses dummy users who do not exist in the provider. The providers treat the dummy users as if they actually exist. By using the dummy users, it can anonymize the personal information without disclosing the users' presence.

%We evaluate the security of the proposed protocol and proof that the protocol does not disclose any sensitive information. In addition, we evaluate the processing and communication cost of the protocol. The evaluation results show that the cost of the proposed protocol is not much higher than that of the existing protocols. 

%Moreover, we evaluate the utility of the proposed protocol with U.S. Census data and health data. Our evaluation results show that the proposed protocol can anonymize them with lower information loss than the existing distributed anonymization method.

%It is expected that our method combine not only census data and health data but also several types of the personal information and there is a possibility that a new service will be created.
%\\

\thispagestyle{plain}
\cleardoublepage
%
%
%%%%%%%% 和文の概要 %%%%%%%%%%%%

\begin{center}
    \Large
    \textbf{SNS に投稿された音声を含む偽情報に対応した早期検出手法の提案}
        
    \vspace{0.4cm}
    \textbf{栁 裕太}
       
    \vspace{0.9cm}
    \textbf{概要}
   
\end{center}

\selectlanguage{japanese} 
%\begin{abstract}
%背景
ソーシャル・ネットワーキング・サービス(SNS)の発展によって誰もが情報を共有できるようになった反面、悪意によって意図的に作成された偽情報が多くの社会問題を引き起こしている。
人手による偽情報の検出には時間がかかるため、人工知能によって自動で検出する手法が求められている。
その中で、利用者の反応を考慮した研究も提案されているが、早期検出を目指した場合に十分な反応を得ることができないという課題がある。
また、新しい情報生成技術の登場や広く使われるSNSの変化によって、検出において必要な要求が大きく広がりつつある。
%よって、文章を中心とした既存の手法では検出性能を長い期間維持しにくいほか、新しい媒体を使用したSNSへの適用も難しい。
具体的には、文章や画像といった情報の形式も、動画や音声といった新たな形式への対応が求められている。
特に音声では、手軽に自然な音声を得られるようになったため、音声を使った偽情報が急増しているにも関わらず、
十分に有効な検出手法が提案されていない。
そこで本論文では現状特に対策が必要な、偽情報を喧伝するなりすまし音声に対する新たな検出手法を2つ提案する。
1つはどの形式の情報・投稿に対しても自然言語として投稿されるコメントを扱う手法であり、もう1つは偽情報を喧伝するなりすまし音声に対して合成/本人と事実/虚偽という2つの観点から検出する手法である。

%実験1
%%目的
音声を含めたどの情報形式へ対応可能な手法として、コメントの生成による検出の補助を目指した。
事実に基づく情報に比べて、偽情報のコメントは批判的な内容になりやすいという点が、先行研究によって明かされている。
よって記事に加えて寄せられたコメントを使った検出が有用である反面、
早期検出においては多数のコメントの活用が難しい部分に着目した。
そこで本研究は早期検出と高精度の両立を目指して、記事に対するコメントを生成して拡散後の状況を擬似的に作り、偽情報検出を補助する手法を提案する。
%手法
提案手法は記事の本文と3件のコメントからコメントを生成する学習を行い、記事の内容に対してどのようなコメントが寄せられたか把握させる。
%この手法の開発にあたって、偽情報を生成する既存手法を改変して実装した。
%実験
%\newpage
英文記事とコメントのデータセットを使用した実験では、学習させた手法に対して本文と2件のコメントから追加で1件コメントを生成させた。
検出性能を調べるため、生成コメントを加えたセットを別の偽情報検出モデルに入力して性能に変化があったか調べた。
比較対象として、記事本文のみで検出を行った場合と、本文と実在コメント2件のみで検出を行った場合を用意した。
提案手法の芸能関連の記事に対する再現率(Recall)は0.695と全体で最も高い結果を示し、より多くの偽情報を検出した。
一方で適合率(Precision)に改善の余地を残しているものの、生成コメントの追加によって検出結果がFakeに覆った割合が48\%と、生成コメントの追加による効果がみられた。
また、政治関連の記事に対してはいずれの指標においても生成コメントを追加しない場合の結果を上回った。
% 既存手法で不足している指摘
% なぜこの手法を開発することにしたか?という部分は確実に聞かれる
% コメント生成→媒体に左右されにくい+何らかのポイントを追加したい
% 音声→現状で手薄
% 全体図で手薄な部分の指摘と全体に適用可能な形式
% サーベイ論文で既にある指摘なら引用して説得力補強
% もう一息説得力がほしい
% 動画対象はどこまでできている?

%\newpage
%実験2
%%目的
また、合成音声が話す偽情報に対して「誰が」「何を」話しているかという2視点から検出を行う手法を提案する。
新たなSNSとして文章や画像以外を扱う形式が出現していることと、生成技術の発展で任意の人物の音声を容易に再現できるようになっている点
%、そして合成技術の発展によって音声波形のみから検出する手法の性能維持が難しくなっている点
から、本研究では音声を対象とした。
%手法
既存の合成音声検出手法の多くは生体認証システムを想定した形式であるため、音声波形を分析する形が中心だった。
しかしながら、SNS上にて合成音声による偽情報を検出したい場合、内容の疑わしさも考慮に入れる必要がある。
よって本研究は虚偽の主張を行うなりすまし音声の検出を目標に、発話内容と音声波形の両面から検出を行う手法を提案する.
提案手法では、既存手法と同じく音声波形を直接入力とする部分と、
発話内容の信憑性を評価するための文章埋め込みを入力とする部分からなる.
%実験
SNS上に投稿された文章を読み上げる合成音声を使った実験では、
音声波形のみを考慮した手法では等価エラー率が50.7\%とランダム選択に近しい結果を示したことに対して、提案手法では17.6\%と改善がみられた。

%結論
これらの提案手法を用いることによって,
疑わしい主張を行う音声に対してコメントの生成という新しい形式の真偽判断材料を提供するとともに,
文章による記事に限らず,波形と内容の両面から真偽確認を可能にすることで,
SNS利用の安全性向上への貢献が期待できる.
%
%\end{abstract}

\cleardoublepage