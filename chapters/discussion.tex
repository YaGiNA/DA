\chapter{考察}
\label{ch:discussion}
本章では、本研究を実社会へ適用する際に想定される効果及び課題について記述する。

\section{多様化する媒体への対応}
\cref{ch:gen_com}では、記事に予想されるコメントを生成することで検出を補助するモデルを提案した。
本モデルの利点として、他の媒体への拡張が容易である点が挙げられる。
コメントはどの媒体の情報に対して共通して自然言語として使われるため、投稿とコメントがセットで提供されているデータセットがあれば、画像・動画・音声問わずどの形式の偽情報にも適用が可能となる。

\section{大規模言語モデルによる影響}
大規模言語モデル(LLM)の性質として、事実に基づかない情報を提供するハルシネーション(Hallucination)が指摘されている \cite{Alkaissi2023-bo}。
これはモデルの学習及び推論において文章の自然さが優先され、内容が事実に基づくかがあまり確認されていない点が理由として考えられる。
よって、LLMのハルシネーションが偽情報検出において意図しない動作を行う危険性がある。

よって、偽情報提供側によるLLMを利用して意図的に偽情報を投稿した場合への対策も必要である。
\cref{ch:rel_res}の通り、既にThai Leら \cite{9338282}によってコメントを考慮した検出モデルを誤認させる手法が提案されている。
よって、コメント生成学習において投稿されたコメントが実際に利用者によって書かれたか考慮する必要が生じている。

実際に、生成された文章をLLMそのものに検出させる研究 \cite{10.1145/3655103.3655106} のほか、
知識ベースをLLMに組み込むことで改善を目指した研究も紹介されている \cite{10.1145/3512467}。
しかしながら、LLMの発展によって人間による文章かAIによる文章か見分けが難しくなっている点も指摘されている \cite{Elkhatat2023,chen2023can}点から、
自然言語処理における生成文章検出タスクのさらなる発展が求められている。


\section{動画への個別対応}
\cref{ch:introduction}の通り、本研究では偽情報動画の流布が他媒体と比べて多くないとして対象から除外した。
一方で、動画生成技術も発展を続けており、Sora \cite{videoworldsimulators2024}に代表されるような高精細な映像も生成可能になりつつある。
開発したOpenAIは安全対策としてC2PAと呼ばれる電子透かし \cite{C2PA}の導入や、
生成プロンプトの内容によって偽情報生成を防止するシステム \cite{AI_2023}の適用を約束している \cite{AI_2024}。

しかしながら動画に限らず生成AI全体において、既存の防止システムの突破(jailbreak)を目的とした研究が幾つかある \cite{NEURIPS2023_fd661313,shayegani2024jailbreak}ため、
偽情報生成・検出と同様いたちごっこの様相を呈している。
また、今後ローカル環境での生成が可能になった場合はシステムによる抑止が難しい点からも、動画生成側の技術発展による偽動画による偽情報投稿が今後急激に増える点が予想される。
検出においては、データセット作成において生成例が必要である点から実際の動画生成モデルの提供を待たなければならないが、今後個別対応が必要と考える。

\section{プラットフォームへの依存}
SNS上の検出を目指した場合、SNSプラットフォームへの依存は不可避の要素である。
しかしながら、プラットフォーム側の姿勢の変化により研究の障壁が大幅に変動する問題がある。
特に$\mathbb{X}$(旧Twitter、本項では$\mathbb{X}$と表記)はイーロン・マスク氏による買収前は研究目的であれば投稿の取得が無償で行えたものの、
本論文執筆現在(2024年3月)は最低でも月額課金として1万投稿/月に100米ドル、100万投稿/月に5000米ドル \cite{Twitter}と高額な料金が必要である。

一方で欧州連合によるデジタルサービス法(Digital Service Act)の第40条(Article 40)によって、欧州の研究機関に対しプラットフォームは非営利の研究目的において無償で投稿の提供が求められている \cite{DSA}。
実際に、2023年12月に欧州委員会より$\mathbb{X}$に対して投稿データ提供に応じるよう要請が出されている \cite{order23}。
しかしながら、$\mathbb{X}$が研究者からの申し出を拒否して料金を要求したという報告 \cite{xreject}もあり、依然状況に変化が見られない。
このようなプラットフォーム側の姿勢によって、偽情報対策を目指す活動が抑制され、研究の裾野が狭まる点を危惧している。

また、買収後にユーザの投稿への閲覧数に対する広告収益配分プログラムを開始した。
より多く投稿に対する注目が集まれば収益を得られるシステムであるため、
偽情報がより拡散されやすくなった点が指摘されている \cite{Davis_2023}。