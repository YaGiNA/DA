\chapter{おわりに}\label{ch:conclusion}
%かつてはマスメディアが中心となって情報を広めていたため、組織内での精査によって発信される情報の質はある程度担保されていた。
%その中で
ソーシャル・ネットワーキング・サービス(SNS)は誰もが様々な情報を発信できるだけではなく、見かけた情報を不特定多数へ共有できることで、広く利用者に受け入れられた。
しかしながら、SNSプラットフォームや有識者による精査に対して投稿された情報が共有によって短い時間によって広く拡散される点から、事実と異なるように意図的に作成された偽情報も広まりやすくなった。

既存手法が抱える問題点として、情報の媒体変化への対応に難しい点が挙げられる。
%例えば自然言語で書かれた偽情報そのものの内容のみで学習した場合、
特に近年は動画や音声といった新しい媒体を主とするSNSも若年層を中心に使われている一方で、
既存の自然言語で書かれた偽情報を想定したモデルはそのような媒体には対応できない。
またもう1つの問題点として、とりわけ音声は任意の人物の音声を短い時間のサンプルで忠実に再現できる一方で、
既存手法による対策は音声波形のみを対象としており、生成側の発展に追いついていない点も挙げられる。
本研究は、それぞれの問題に対応する新たな手法を提案する。

媒体変化に頑強な手法として、あらゆる形式に対して共通して自然言語の形をとるコメントを活用した。
既に既存手法で偽情報に対してコメントが批判的になるため検出に有用であることが示されていた。
一方で拡散の初期段階で検出を目指す場合、使用できるコメントの数は制限される。
本研究では検出を補助するために、コメントの生成を行った。
実装にあたって、架空の記事を生成する手法を改変し、実際に投稿された記事と3件のコメントから生成の学習を行った。
実験では、まず実際に投稿された記事と2件の投稿済みコメントから学習済生成モデルがコメントを追加した。
続いて、別の偽情報検出モデルにこれらの情報を入力し、偽情報であるかの判断を行った。
検出の結果は、コメントそのものの有用性の確認として記事本文のみで検出を行った場合と、コメント生成の有用性の確認として記事と投稿済コメント2件で検出を行った場合と比較した。
提案手法の偽情報全体の件数のうち実際に検出できた割合である再現率(Recall)は0.695と全体で最も高い結果を示した。
%一方で偽情報と検出した中で実際に偽情報だった割合を示す適合率(Precision)には改善の余地を残した。
また投稿済みコメントのみで分類したときに事実に基づく情報と誤って判断した件数のうち、生成コメントの追加によって偽情報と正しく判断できた割合が48\%と、コメント生成の効果を確認した。
この手法は必ず自然言語の形を取るコメントの生成という新しい形式の真偽判断材料を提供する。
これはSNSプラットフォームの変化に頑強であるため、長い期間利用者や検出モデルに対して真偽を判断する新たな情報を与えられる。

音声による偽情報対策として、発話された内容も考慮して偽情報の検出も目指した。
これは既存手法にはみられない新しい視点として、SNS上で投稿された音声による偽情報の検出を目指すには発話内容の考慮を取り入れている。
具体的には、音声波形の分析を行う既存手法に加えて、文章埋め込みとして話された内容から得た情報から信憑性を評価する部分から構成している。
実験では、実際にSNSに投稿された偽情報文章から直近に提案された音声合成手法によって得た音声データセットを使用した。
音声波形のみを考慮した既存手法では、他人受入率と本人拒否率が一致するよう閾値を調整した際のエラー率である等価エラー率が50.7\%を示し、無作為に真偽を判断した場合に近しい結果を示した。
一方で提案手法では等価エラー率が44.6\%まで改善し、提案する視点の有用性が示された。
この手法は現在検出が難しい疑わしい主張を行う音声に対して音声波形と発話内容から信憑性を評価できるため、
SNS上に投稿された偽情報音声を正確に検出できる。

今後SNS利用の安全性を維持するためには、共有される情報の媒体の多様化に対応しなければならない。
本研究が提案する2つの手法は、それぞれ媒体の多様化に頑強であることと、現状さらなる対策が必要な媒体へ特化しているという特長がある。
